\RequireXeTeX

\RequirePackage{xcolor}
% From https://www.dtu.dk/upload/dtu%20kommunikation/designguide/designplatform_farver2010.pdf
%
% Primally colors
\definecolor{dtured}{cmyk}{0,.91,.72,.23}
\definecolor{dtugray}{cmyk}{0,0,0,.56}
% Secondary colors
\definecolor{s12}{cmyk}{0,0.25,1,0}    % yellow
\definecolor{s01}{cmyk}{0,0.5,1,0}     %
\definecolor{s02}{cmyk}{0,0.75,1,0}    % orange
\definecolor{s03}{cmyk}{0,1,1,0}       % red
\definecolor{s04}{cmyk}{0,1,1,0.5}     %
\definecolor{s05}{cmyk}{.0,1,.0,.0}    % magenta
\definecolor{s06}{cmyk}{.25,1,.0,.0}   %
\definecolor{s07}{cmyk}{.25,1,0,0}     % purple
\definecolor{s08}{cmyk}{.75,1,0,0}     %
\definecolor{s09}{cmyk}{.75,.75,.0,.0} %
\definecolor{s13}{cmyk}{.75,.50,.0,.0} % blue
\definecolor{s10}{cmyk}{.5,0,0,0}      %
\definecolor{s11}{cmyk}{.5,.0,.0,.0}   %
\definecolor{s14}{cmyk}{.5,0,1,0}      % green
% Tinted colors
\definecolor{s14a}{cmyk}{.6,0,1,.25}   % green

% Text fonts (http://www.macfreek.nl/memory/Fonts_in_LaTeX)
% Install fonts from /usr/local/texlive/<version>/texmf-dist/fonts/opentype/public
\usepackage{fontspec}

% Sans-serif font
\setsansfont[
    Ligatures=TeX,
    Extension=.otf,
    UprightFont=*-regular,
    BoldFont=*-bold,
    ItalicFont=*-italic,
    BoldItalicFont=*-bolditalic,
    %SlantedFont=,
    %BoldSlantedFont=,
    %SmallCapsFont=
    Scale=0.8      % Adjustmens when using math in sections
]{texgyreadventor}
%\setsansfont[Ligatures=TeX]{Neo Sans Intel}    % Neo Sans Intel – Like DTU font but more symbols
%\setsansfont[
%    Ligatures=TeX,
%    Scale=0.8
%]{NeoSans}           % NeoSans – DTU font (missing `+' symbols and other)
%\setsansfont[Ligatures=TeX]{CMU Sans Serif}    % Computer Modern Unicode font
%\setsansfont[Ligatures=TeX]{Latin Modern Sans} % Latin Modern Sans serif font

% Use this for more convienent sans serif font in math mode.
%\setmathsf{Latin Modern Sans}

\usepackage{eso-pic}                  % Watermark and other bag

\showtrimsoff
\papersizeswitch{b5paper}{
    % Stock and paper layout
    \pagebv
    \setlrmarginsandblock{26mm}{20mm}{*}
    \setulmarginsandblock{35mm}{30mm}{*}
    \setheadfoot{8mm}{10mm}
    \setlength{\headsep}{7mm}
    \setlength{\marginparwidth}{18mm}
    \setlength{\marginparsep}{2mm}
}{
    \papersizeswitch{a4paper}{
        \pageaiv
        \setlength{\trimtop}{0pt}
        \setlength{\trimedge}{\stockwidth}
        \addtolength{\trimedge}{-\paperwidth}
        \settypeblocksize{634pt}{448.13pt}{*}
        \setulmargins{4cm}{*}{*}
        \setlrmargins{*}{*}{0.66}
        \setmarginnotes{17pt}{51pt}{\onelineskip}
        \setheadfoot{\onelineskip}{2\onelineskip}
        \setheaderspaces{*}{2\onelineskip}{*}
    }{
    }
}
\ifnum\strcmp{\showtrims}{true}=0
    % For printing B5 on A4 with trimmarks
    \showtrimson
    \papersizeswitch{b5paper}{\stockaiv}{\stockaiii}
    \setlength{\trimtop}{\stockheight}
    \addtolength{\trimtop}{-\paperheight}
    \setlength{\trimtop}{0.5\trimtop}
    \setlength{\trimedge}{\stockwidth}
    \addtolength{\trimedge}{-\paperwidth}
    \setlength{\trimedge}{0.5\trimedge}

    % bigger todos if trim marks
    \setmarginnotes{10pt}{95pt}{\onelineskip}

    \trimLmarks

    % put jobname in left top trim mark
    \renewcommand*{\tmarktl}{%
      \begin{picture}(0,0)
        \unitlength 1mm
        \thinlines
        \put(-2,0){\line(-1,0){18}}
        \put(0,2){\line(0,1){18}}
        \put(3,15){\normalfont\ttfamily\fontsize{8bp}{10bp}\selectfont\jobname\ \
          \today\ \
          \printtime\ \
          Page \thepage}
      \end{picture}}

    % Remove middle trim marks for cleaner layout
    \renewcommand*{\tmarktm}{}
    \renewcommand*{\tmarkml}{}
    \renewcommand*{\tmarkmr}{}
    \renewcommand*{\tmarkbm}{}
\fi

% Todos
\usepackage{totcount}                 % For total counting of counters
\def\todoshowing{}
\ifnum\strcmp{\showtodos}{false}=0
    \def\todoshowing{disable}
\fi
\usepackage[colorinlistoftodos,\todoshowing]{todonotes} % Todonotes package for nice todos
\newtotcounter{todocounter}           % Creates counter in todo
\let\oldtodo\todo
\newcommand*{\newtodo}[2][]{\stepcounter{todocounter}\oldtodo[#1]{\thesection~(\thetodocounter)~#2}}
\let\todo\newtodo
\let\oldmissingfigure\missingfigure
\newcommand*{\newmissingfigure}[2][]{\stepcounter{todocounter}\oldmissingfigure[#1]{\thesection~(\thetodocounter)~#2}}
\let\missingfigure\newmissingfigure
\makeatletter
\newcommand*{\mylistoftodos}{% Only show list if there are todos
\if@todonotes@disabled
\else
    \ifnum\totvalue{todocounter}>0
        \markboth{\@todonotes@todolistname}{\@todonotes@todolistname}
        \phantomsection\todototoc
        \listoftodos
    \else
    \fi
\fi
}
\makeatother
\newcommand{\lesstodo}[2][]{\todo[color=green!40,#1]{#2}}
\newcommand{\moretodo}[2][]{\todo[color=red!40,#1]{#2}}

% Chapterstyle
\makeatletter
\makechapterstyle{mychapterstyle}{
    \chapterstyle{default}
    \def\format{\normalfont\sffamily}

    \setlength\beforechapskip{0mm}

    \renewcommand*{\chapnamefont}{\format\fontsize{32}{32}}
    \renewcommand*{\chapnumfont}{\format\fontsize{32}{32}\selectfont}
    \renewcommand*{\chaptitlefont}{\format\fontsize{24}{24}\selectfont}

    \renewcommand*{\printchaptername}{\chapnamefont\MakeUppercase{\@chapapp}}
    \patchcommand{\printchaptername}{\begingroup\color{dtugray}}{\endgroup}
    \renewcommand*{\chapternamenum}{\space\space}
    \patchcommand{\printchapternum}{\begingroup\color{dtured}}{\endgroup}
    \renewcommand*{\printchapternonum}{%
        \vphantom{\printchaptername\chapternamenum\chapnumfont 1}
        \afterchapternum
    }

    \setlength\midchapskip{1ex}

    \renewcommand*{\printchaptertitle}[1]{\raggedleft \chaptitlefont ##1}
    \renewcommand*{\afterchaptertitle}{\vskip0.5\onelineskip \hrule \vskip1.3\onelineskip}
}
\makeatother
\chapterstyle{mychapterstyle}

% Header and footer
\def\hffont{\sffamily\small}
\makepagestyle{myruled}
\makeheadrule{myruled}{\textwidth}{\normalrulethickness}
\makeevenhead{myruled}{\hffont\thepage}{}{\hffont\leftmark}
\makeoddhead{myruled}{\hffont\rightmark}{}{\hffont\thepage}
\makeevenfoot{myruled}{}{}{}
\makeoddfoot{myruled}{}{}{}
\makepsmarks{myruled}{
    \nouppercaseheads
    \createmark{chapter}{both}{shownumber}{}{\space}
    \createmark{section}{right}{shownumber}{}{\space}
    \createplainmark{toc}{both}{\contentsname}
    \createplainmark{lof}{both}{\listfigurename}
    \createplainmark{lot}{both}{\listtablename}
    \createplainmark{bib}{both}{\bibname}
    \createplainmark{index}{both}{\indexname}
    \createplainmark{glossary}{both}{\glossaryname}
}
\pagestyle{myruled}
\copypagestyle{cleared}{myruled}      % When \cleardoublepage, use myruled instead of empty
\makeevenhead{cleared}{\hffont\thepage}{}{} % Remove leftmark on cleared pages

\makeevenfoot{plain}{}{}{}            % No page number on plain even pages (chapter begin)
\makeoddfoot{plain}{}{}{}             % No page number on plain odd pages (chapter begin)

% \*section, \*paragraph font styles
\setsecheadstyle              {\huge\sffamily\raggedright}
\setsubsecheadstyle           {\LARGE\sffamily\raggedright}
\setsubsubsecheadstyle        {\Large\sffamily\raggedright}
%\setparaheadstyle             {\normalsize\sffamily\itseries\raggedright}
%\setsubparaheadstyle          {\normalsize\sffamily\raggedright}


% Hack to make right pdfbookmark link. The normal behavior links just below the chapter title. This hack put the link just above the chapter like any other normal use of \chapter.
% Another solution can be found in http://tex.stackexchange.com/questions/59359/certain-hyperlinks-memoirhyperref-placed-too-low
\makeatletter
\renewcommand{\@memb@bchap}{%
  \ifnobibintoc\else
    \phantomsection
    \addcontentsline{toc}{chapter}{\bibname}%
  \fi
  \chapter*{\bibname}%
  \bibmark
  \prebibhook
}
\let\oldtableofcontents\tableofcontents
\newcommand{\newtableofcontents}{
    \@ifstar{\oldtableofcontents*}{
        \phantomsection\addcontentsline{toc}{chapter}{\contentsname}\oldtableofcontents*}}
\let\tableofcontents\newtableofcontents
\makeatother

% Confidential
\newcommand{\confidentialbox}[1]{
    \put(0,0){\parbox[b][\paperheight]{\paperwidth}{
        \begin{vplace}
            \centering
            \scalebox{1.3}{
                \begin{tikzpicture}
                    \node[very thick,draw=red!#1,color=red!#1,
                          rounded corners=2pt,inner sep=8pt,rotate=-20]
                          {\sffamily \HUGE \MakeUppercase{Confidential}};
                \end{tikzpicture}
            }
        \end{vplace}
    }}
}

% Prefrontmatter
\newcommand{\prefrontmatter}{
    \pagenumbering{alph}
    \ifnum\strcmp{\confidential}{true}=0
        \AddToShipoutPictureBG{\confidentialbox{10}}   % 10% classified box in background on each page
        \AddToShipoutPictureFG*{\confidentialbox{100}} % 100% classified box in foreground on first page
    \fi
}

% DTU frieze
\newcommand{\frieze}{%
    \AddToShipoutPicture*{
        \put(0,0){
            \parbox[b][\paperheight]{\paperwidth}{%
                \includegraphics[trim=130mm 0 0 0,width=0.9\textwidth]{Graphic/Titlepage/DTU Friese}
                \vspace*{2.5cm}
            }
        }
    }
}

% This is a double sided book. If there is a last empty page lets use it for some fun e.g. the frieze.
% NB: For a fully functional hack the \clearpage used in \include does some odd thinks with the sequence numbering. Thefore use \input instead of \include at the end of the book. If bibliography is used at last everything should be ok.
\makeatletter
% Adjust so gatherings is allowd for single sheets too! (hacking functions in memoir.dtx)
\patchcmd{\leavespergathering}{\ifnum\@memcnta<\tw@}{\ifnum\@memcnta<\@ne}{
    \leavespergathering{1}
    % Insert the frieze
    \patchcmd{\@memensuresigpages}{\repeat}{\repeat\frieze}{}{}
}{}
\makeatother
