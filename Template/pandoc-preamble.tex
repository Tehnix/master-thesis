%%%%%%%%%%%%%%%%%%%%%%%%%%%%%%%%%%%%%%%%%%%%%%%%%%%%%%%%%%%%%% Static variables.
\date{$date$}

$if(author)$
\author{$for(author)$$author$$sep$ \and $endfor$}
\def\thesisauthor{$for(author)$ $author$$sep$ \\ $endfor$}
$endif$
$if(title)$
\title{$title$}
\def\thesistitle{$title$}
$endif$
\providecommand{\subtitle}[1]{}
$if(subtitle)$
\subtitle{$subtitle$}
\def\thesissubtitle{$subtitle$}
$else$
\subtitle{}
\def\thesissubtitle{}
$endif$

$if(thesis.abbreviation)$
\def\thesistypeabbr{$thesis.abbreviation$}
$else$
\def\thesistypeabbr{}
$endif$
$if(thesis.type)$
\def\thesistype{$thesis.type$}
$else$
\def\thesistype{}
$endif$
$if(thesis.type)$
\def\thesislocation{$thesis.location$}
$else$
\def\thesislocation{}
$endif$


$if(thesis.institute)$
\providecommand{\institute}[1]{}
\institute{$for(thesis.institute)$$thesis.institute$$sep$ \and $endfor$}
\def\thesisinstitute{$for(thesis.institute)$$thesis.institute$$sep$ \and $endfor$}
$else$
\institute{}
\def\thesisinstitute{}
$endif$
$if(thesis.institute-longname)$
\def\thesisinstitutelongname{$thesis.institute-longname$}
$else$
\def\thesisinstitutelongname{}
$endif$
$if(thesis.university)$
\def\thesisuniversity{$thesis.university$}
$else$
\def\thesisuniversity{}
$endif$
$if(thesis.address)$
\def\thesisaddress{$for(thesis.address)$$thesis.address$$sep$ \\ $endfor$}
$else$
\def\thesisaddress{}
$endif$

$if(papersize)$
\def\papersize{$papersize$} % Final papersize (b5paper/a4paper), recommended papersize for DTU Compute is b5paper
$else$
\def\papersize{b5paper}
$endif$
\def\showtrims{false} % Print on larger paper than \papersize and show trim marks (true/false)?
$if(thesis.todos)$
\def\showtodos{$thesis.todos$}  % Show todos (true/false)?
$else$
\def\showtodos{false}  % Show todos (true/false)?
$endif$
$if(thesis.confidential)$
\def\confidential{$thesis.confidential$}  % Show todos (true/false)?
$else$
\def\confidential{false}  % Show todos (true/false)?
$endif$


%%%%%%%%%%%%%%%%%%%%%%%%%%%%%%%%%%%%%%%%%%%%%%%%%%%%%%%%%%%%%%%%%% Pre-preample.
\documentclass[$if(fontsize)$$fontsize$,$endif$$if(lang)$$babel-lang$,$endif$$if(papersize)$$papersize$paper,$endif$$for(classoption)$$classoption$$sep$,$endfor$,twoside,showtrims,extrafontsizes]{$documentclass$}

\RequirePackage{xcolor}
% From https://www.dtu.dk/upload/dtu%20kommunikation/designguide/designplatform_farver2010.pdf
%
% Primally colors
\definecolor{dtured}{cmyk}{0,.91,.72,.23}
\definecolor{dtugray}{cmyk}{0,0,0,.56}
% Secondary colors
\definecolor{s12}{cmyk}{0,0.25,1,0}    % yellow
\definecolor{s01}{cmyk}{0,0.5,1,0}     %
\definecolor{s02}{cmyk}{0,0.75,1,0}    % orange
\definecolor{s03}{cmyk}{0,1,1,0}       % red
\definecolor{s04}{cmyk}{0,1,1,0.5}     %
\definecolor{s05}{cmyk}{.0,1,.0,.0}    % magenta
\definecolor{s06}{cmyk}{.25,1,.0,.0}   %
\definecolor{s07}{cmyk}{.25,1,0,0}     % purple
\definecolor{s08}{cmyk}{.75,1,0,0}     %
\definecolor{s09}{cmyk}{.75,.75,.0,.0} %
\definecolor{s13}{cmyk}{.75,.50,.0,.0} % blue
\definecolor{s10}{cmyk}{.5,0,0,0}      %
\definecolor{s11}{cmyk}{.5,.0,.0,.0}   %
\definecolor{s14}{cmyk}{.5,0,1,0}      % green
% Tinted colors
\definecolor{s14a}{cmyk}{.6,0,1,.25}   % green

% Text fonts (http://www.macfreek.nl/memory/Fonts_in_LaTeX)
% Install fonts from /usr/local/texlive/<version>/texmf-dist/fonts/opentype/public
\usepackage{fontspec}

% Sans-serif font
\setsansfont[
    Ligatures=TeX,
    Extension=.otf,
    UprightFont=*-regular,
    BoldFont=*-bold,
    ItalicFont=*-italic,
    BoldItalicFont=*-bolditalic,
    %SlantedFont=,
    %BoldSlantedFont=,
    %SmallCapsFont=
    Scale=0.8      % Adjustmens when using math in sections
]{texgyreadventor}
%\setsansfont[Ligatures=TeX]{Neo Sans Intel}    % Neo Sans Intel – Like DTU font but more symbols
%\setsansfont[
%    Ligatures=TeX,
%    Scale=0.8
%]{NeoSans}           % NeoSans – DTU font (missing `+' symbols and other)
%\setsansfont[Ligatures=TeX]{CMU Sans Serif}    % Computer Modern Unicode font
%\setsansfont[Ligatures=TeX]{Latin Modern Sans} % Latin Modern Sans serif font

% Use this for more convienent sans serif font in math mode.
%\setmathsf{Latin Modern Sans}


$if(beamerarticle)$
\usepackage{beamerarticle} % needs to be loaded first
$endif$

$if(fontfamily)$
\usepackage[$for(fontfamilyoptions)$$fontfamilyoptions$$sep$,$endfor$]{$fontfamily$}
$else$
\usepackage{lmodern}
$endif$

$if(linestretch)$
\usepackage{setspace}
\setstretch{$linestretch$}
$endif$

\usepackage{amssymb,amsmath}
\usepackage{ifxetex,ifluatex}
\usepackage{fixltx2e} % provides \textsubscript
\ifnum 0\ifxetex 1\fi\ifluatex 1\fi=0 % if pdftex
  \usepackage[$if(fontenc)$$fontenc$$else$T1$endif$]{fontenc}
  \usepackage[utf8]{inputenc}
$if(euro)$
  \usepackage{eurosym}
$endif$
\else % if luatex or xelatex
  \ifxetex
    \usepackage{fontspec}
  \else
    \usepackage{fontspec}
  \fi
  \defaultfontfeatures{Ligatures=TeX,Scale=MatchLowercase}
\fi

$for(fontfamilies)$
\newfontfamily{$fontfamilies.name$}[$fontfamilies.options$]{$fontfamilies.font$}
$endfor$

$if(euro)$
\newcommand{\euro}{€}
$endif$
$if(mainfont)$
\setmainfont[$for(mainfontoptions)$$mainfontoptions$$sep$,$endfor$]{$mainfont$}
$endif$
$if(sansfont)$
\setsansfont[$for(sansfontoptions)$$sansfontoptions$$sep$,$endfor$]{$sansfont$}
$endif$
$if(monofont)$
\setmonofont[Mapping=tex-ansi$if(monofontoptions)$,$for(monofontoptions)$$monofontoptions$$sep$,$endfor$$endif$]{$monofont$}
$endif$
$if(mathfont)$
\setmathfont(Digits,Latin,Greek)[$for(mathfontoptions)$$mathfontoptions$$sep$,$endfor$]{$mathfont$}
$endif$
$if(CJKmainfont)$
\usepackage{xeCJK}
\setCJKmainfont[$for(CJKoptions)$$CJKoptions$$sep$,$endfor$]{$CJKmainfont$}
$endif$

% use upquote if available, for straight quotes in verbatim environments
\IfFileExists{upquote.sty}{\usepackage{upquote}}{}
% use microtype if available
\IfFileExists{microtype.sty}{%
\usepackage[$for(microtypeoptions)$$microtypeoptions$$sep$,$endfor$]{microtype}
\UseMicrotypeSet[protrusion]{basicmath} % disable protrusion for tt fonts
}{}
\PassOptionsToPackage{hyphens}{url} % url is loaded by hyperref

$if(verbatim-in-note)$
\usepackage{fancyvrb}
$endif$

\usepackage[unicode=true]{hyperref}

$if(colorlinks)$
\PassOptionsToPackage{usenames,dvipsnames}{color} % color is loaded by hyperref
$endif$

\hypersetup{
            pdfauthor={\thesisauthor{}},
            pdftitle={\thesistitle{}},
            pdfsubject={\thesissubtitle{}},
$if(title-meta)$
            pdftitle={$title-meta$},
$endif$
$if(author-meta)$
            pdfauthor={$author-meta$},
$endif$
$if(keywords)$
            pdfkeywords={$for(keywords)$$keywords$$sep$, $endfor$},
$endif$
$if(colorlinks)$
            colorlinks=true,
            linkcolor=$if(linkcolor)$$linkcolor$$else$Maroon$endif$,
            citecolor=$if(citecolor)$$citecolor$$else$Blue$endif$,
            urlcolor=$if(urlcolor)$$urlcolor$$else$Blue$endif$,
$else$
            colorlinks=false,
$endif$
            pdfdisplaydoctitle,
            bookmarksnumbered=true,
            bookmarksopen,
            breaklinks,
            linktoc=all,
            plainpages=false,
            unicode=true,
            citebordercolor=dtured,           % color of links to bibliography
            filebordercolor=dtured,           % color of file links
            linkbordercolor=dtured,           % color of internal links (change box color with linkbordercolor)
            urlbordercolor=s13,               % color of external links
            hidelinks,                        % Do not show boxes or colored links.
}
\urlstyle{same}  % don't use monospace font for urls

$if(verbatim-in-note)$
\VerbatimFootnotes % allows verbatim text in footnotes
$endif$

$if(geometry)$
\usepackage[$for(geometry)$$geometry$$sep$,$endfor$]{geometry}
$endif$

$if(lang)$
\ifnum 0\ifxetex 1\fi\ifluatex 1\fi=0 % if pdftex
  \usepackage[shorthands=off,$for(babel-otherlangs)$$babel-otherlangs$,$endfor$main=$babel-lang$]{babel}
$if(babel-newcommands)$
  $babel-newcommands$
$endif$
\else
  \usepackage{polyglossia}
  \setmainlanguage[$polyglossia-lang.options$]{$polyglossia-lang.name$}
$for(polyglossia-otherlangs)$
  \setotherlanguage[$polyglossia-otherlangs.options$]{$polyglossia-otherlangs.name$}
$endfor$
\fi
$endif$

$if(natbib)$
\usepackage{natbib}
\bibliographystyle{$if(biblio-style)$$biblio-style$$else$plainnat$endif$}
$endif$

$if(biblatex)$
\usepackage[backend=biber,
            % alldates=long,
            % backref=true,
            % abbreviate=false,
            % dateabbrev=false,
            style=alphabetic]{biblatex}
$for(bibliography)$
\addbibresource{$bibliography$}
$endfor$
$endif$

$if(listings)$
\usepackage{listings}
$endif$

$if(lhs)$
\lstnewenvironment{code}{\lstset{language=Haskell,basicstyle=\small\ttfamily}}{}
$endif$

$if(highlighting-macros)$
$highlighting-macros$
$endif$

$if(tables)$
\usepackage{longtable,booktabs}
% Fix footnotes in tables (requires footnote package)
\IfFileExists{footnote.sty}{\usepackage{footnote}\makesavenoteenv{long table}}{}
$endif$

$if(graphics)$
\usepackage{graphicx,grffile}
\makeatletter
\def\maxwidth{\ifdim\Gin@nat@width>\linewidth\linewidth\else\Gin@nat@width\fi}
\def\maxheight{\ifdim\Gin@nat@height>\textheight\textheight\else\Gin@nat@height\fi}
\makeatother
% Scale images if necessary, so that they will not overflow the page
% margins by default, and it is still possible to overwrite the defaults
% using explicit options in \includegraphics[width, height, ...]{}
%\setkeys{Gin}{width=\maxwidth,height=\maxheight,keepaspectratio}
$endif$

$if(links-as-notes)$
% Make links footnotes instead of hotlinks:
\renewcommand{\href}[2]{#2\footnote{\url{#1}}}
$endif$

$if(strikeout)$
\usepackage[normalem]{ulem}
% avoid problems with \sout in headers with hyperref:
\pdfstringdefDisableCommands{\renewcommand{\sout}{}}
$endif$

$if(indent)$
$else$
\IfFileExists{parskip.sty}{%
\usepackage{parskip}
}{% else
\setlength{\parindent}{0pt}%
\setlength{\parskip}{6pt plus 2pt minus 1pt}
}
$endif$

\setlength{\emergencystretch}{3em}  % prevent overfull lines
\providecommand{\tightlist}{%
  \setlength{\itemsep}{0pt}\setlength{\parskip}{0pt}}

$if(numbersections)$
\setcounter{secnumdepth}{$if(secnumdepth)$$secnumdepth$$else$5$endif$}
$else$
\setcounter{secnumdepth}{0}
$endif$

$if(subparagraph)$
$else$
% Redefines (sub)paragraphs to behave more like sections
\ifx\paragraph\undefined\else
\let\oldparagraph\paragraph
\renewcommand{\paragraph}[1]{\oldparagraph{#1}\mbox{}}
\fi
\ifx\subparagraph\undefined\else
\let\oldsubparagraph\subparagraph
\renewcommand{\subparagraph}[1]{\oldsubparagraph{#1}\mbox{}}
\fi
$endif$

$if(dir)$
\ifxetex
  % load bidi as late as possible as it modifies e.g. graphicx
  $if(latex-dir-rtl)$
  \usepackage[RTLdocument]{bidi}
  $else$
  \usepackage{bidi}
  $endif$
\fi
\ifnum 0\ifxetex 1\fi\ifluatex 1\fi=0 % if pdftex
  \TeXXeTstate=1
  \newcommand{\RL}[1]{\beginR #1\endR}
  \newcommand{\LR}[1]{\beginL #1\endL}
  \newenvironment{RTL}{\beginR}{\endR}
  \newenvironment{LTR}{\beginL}{\endL}
\fi
$endif$

% set default figure placement to htbp
\makeatletter
\def\fps@figure{htbp}
\makeatother

\usepackage{longtable}
\usepackage{caption}
\captionsetup[table]{position=bottom}

$if(loa)$
\usepackage[acronym, toc]{glossaries}
$endif$

\usepackage{verbatim}
\usepackage{cleveref}
