%%%%%%%%%%%%%%%%%%%%%%%%%%%%%%%%%%%%%%%%%%%%%%%%%%%%%%%%%%%%%% Static variables.
\date{2017-01-28}

\author{Christian Kjaer Laustsen \and s124324 @ DTU \and 20176018 @ KAIST}
\def\thesisauthor{ Christian Kjaer Laustsen \\  s124324 @ DTU \\  20176018 @ KAIST}
\title{Exploring the use of purely functional programming languages for
offloading of mobile computations}
\def\thesistitle{Exploring the use of purely functional programming languages for
offloading of mobile computations}
\providecommand{\subtitle}[1]{}
\subtitle{}
\def\thesissubtitle{}

\def\thesistypeabbr{M.Sc.}
\def\thesistype{Master of Science in Engineering}
\def\thesislocation{Kongens Lyngby}


\providecommand{\institute}[1]{}
\institute{DTU Fotonik}
\def\thesisinstitute{DTU Fotonik}
\def\thesisinstitutelongname{Department of Photonics Engineering}
\def\thesisuniversity{Technical University of Denmark}
\def\thesisaddress{Ørsteds Plads \\ Building 343 \\ 2800 Kongens Lyngby, Denmark \\ Phone +45 4525 6352 \\ info@fotonik.dtu.dk \\ www.fotonik.dtu.dk}

\def\papersize{b5paper}
\def\showtrims{false} % Print on larger paper than \papersize and show trim marks (true/false)?
\def\showtodos{false}  % Show todos (true/false)?
\def\confidential{false}  % Show todos (true/false)?


%%%%%%%%%%%%%%%%%%%%%%%%%%%%%%%%%%%%%%%%%%%%%%%%%%%%%%%%%%%%%%%%%% Pre-preample.
\documentclass[10pt,,twoside,showtrims,extrafontsizes]{memoir}

\RequirePackage{xcolor}
% From https://www.dtu.dk/upload/dtu%20kommunikation/designguide/designplatform_farver2010.pdf
%
% Primally colors
\definecolor{dtured}{cmyk}{0,.91,.72,.23}
\definecolor{dtugray}{cmyk}{0,0,0,.56}
% Secondary colors
\definecolor{s12}{cmyk}{0,0.25,1,0}    % yellow
\definecolor{s01}{cmyk}{0,0.5,1,0}     %
\definecolor{s02}{cmyk}{0,0.75,1,0}    % orange
\definecolor{s03}{cmyk}{0,1,1,0}       % red
\definecolor{s04}{cmyk}{0,1,1,0.5}     %
\definecolor{s05}{cmyk}{.0,1,.0,.0}    % magenta
\definecolor{s06}{cmyk}{.25,1,.0,.0}   %
\definecolor{s07}{cmyk}{.25,1,0,0}     % purple
\definecolor{s08}{cmyk}{.75,1,0,0}     %
\definecolor{s09}{cmyk}{.75,.75,.0,.0} %
\definecolor{s13}{cmyk}{.75,.50,.0,.0} % blue
\definecolor{s10}{cmyk}{.5,0,0,0}      %
\definecolor{s11}{cmyk}{.5,.0,.0,.0}   %
\definecolor{s14}{cmyk}{.5,0,1,0}      % green
% Tinted colors
\definecolor{s14a}{cmyk}{.6,0,1,.25}   % green

% Text fonts (http://www.macfreek.nl/memory/Fonts_in_LaTeX)
% Install fonts from /usr/local/texlive/<version>/texmf-dist/fonts/opentype/public
\usepackage{fontspec}

% Sans-serif font
\setsansfont[
    Ligatures=TeX,
    Extension=.otf,
    UprightFont=*-regular,
    BoldFont=*-bold,
    ItalicFont=*-italic,
    BoldItalicFont=*-bolditalic,
    %SlantedFont=,
    %BoldSlantedFont=,
    %SmallCapsFont=
    Scale=0.8      % Adjustmens when using math in sections
]{texgyreadventor}
%\setsansfont[Ligatures=TeX]{Neo Sans Intel}    % Neo Sans Intel – Like DTU font but more symbols
%\setsansfont[
%    Ligatures=TeX,
%    Scale=0.8
%]{NeoSans}           % NeoSans – DTU font (missing `+' symbols and other)
%\setsansfont[Ligatures=TeX]{CMU Sans Serif}    % Computer Modern Unicode font
%\setsansfont[Ligatures=TeX]{Latin Modern Sans} % Latin Modern Sans serif font

% Use this for more convienent sans serif font in math mode.
%\setmathsf{Latin Modern Sans}



\usepackage{lmodern}


\usepackage{amssymb,amsmath}
\usepackage{ifxetex,ifluatex}
\usepackage{fixltx2e} % provides \textsubscript
\ifnum 0\ifxetex 1\fi\ifluatex 1\fi=0 % if pdftex
  \usepackage[T1]{fontenc}
  \usepackage[utf8]{inputenc}
\else % if luatex or xelatex
  \ifxetex
    \usepackage{fontspec}
  \else
    \usepackage{fontspec}
  \fi
  \defaultfontfeatures{Ligatures=TeX,Scale=MatchLowercase}
\fi



% use upquote if available, for straight quotes in verbatim environments
\IfFileExists{upquote.sty}{\usepackage{upquote}}{}
% use microtype if available
\IfFileExists{microtype.sty}{%
\usepackage[]{microtype}
\UseMicrotypeSet[protrusion]{basicmath} % disable protrusion for tt fonts
}{}
\PassOptionsToPackage{hyphens}{url} % url is loaded by hyperref


\usepackage[unicode=true]{hyperref}


\hypersetup{
            pdfauthor={\thesisauthor{}},
            pdftitle={\thesistitle{}},
            pdfsubject={\thesissubtitle{}},
            pdftitle={Exploring the use of purely functional programming languages for offloading of mobile computations},
            pdfauthor={Christian Kjaer Laustsen; s124324 @ DTU; 20176018 @ KAIST},
            colorlinks=false,
            pdfdisplaydoctitle,
            bookmarksnumbered=true,
            bookmarksopen,
            breaklinks,
            linktoc=all,
            plainpages=false,
            unicode=true,
            citebordercolor=dtured,           % color of links to bibliography
            filebordercolor=dtured,           % color of file links
            linkbordercolor=dtured,           % color of internal links (change box color with linkbordercolor)
            urlbordercolor=s13,               % color of external links
            hidelinks,                        % Do not show boxes or colored links.
}
\urlstyle{same}  % don't use monospace font for urls





\usepackage[backend=biber,
            % alldates=long,
            % backref=true,
            % abbreviate=false,
            % dateabbrev=false,
            style=alphabetic]{biblatex}
\addbibresource{Bibliography/Bibliography.bib}



\usepackage{color}
\usepackage{fancyvrb}
\newcommand{\VerbBar}{|}
\newcommand{\VERB}{\Verb[commandchars=\\\{\}]}
\DefineVerbatimEnvironment{Highlighting}{Verbatim}{commandchars=\\\{\}}
% Add ',fontsize=\small' for more characters per line
\newenvironment{Shaded}{}{}
\newcommand{\KeywordTok}[1]{\textcolor[rgb]{0.00,0.44,0.13}{\textbf{#1}}}
\newcommand{\DataTypeTok}[1]{\textcolor[rgb]{0.56,0.13,0.00}{#1}}
\newcommand{\DecValTok}[1]{\textcolor[rgb]{0.25,0.63,0.44}{#1}}
\newcommand{\BaseNTok}[1]{\textcolor[rgb]{0.25,0.63,0.44}{#1}}
\newcommand{\FloatTok}[1]{\textcolor[rgb]{0.25,0.63,0.44}{#1}}
\newcommand{\ConstantTok}[1]{\textcolor[rgb]{0.53,0.00,0.00}{#1}}
\newcommand{\CharTok}[1]{\textcolor[rgb]{0.25,0.44,0.63}{#1}}
\newcommand{\SpecialCharTok}[1]{\textcolor[rgb]{0.25,0.44,0.63}{#1}}
\newcommand{\StringTok}[1]{\textcolor[rgb]{0.25,0.44,0.63}{#1}}
\newcommand{\VerbatimStringTok}[1]{\textcolor[rgb]{0.25,0.44,0.63}{#1}}
\newcommand{\SpecialStringTok}[1]{\textcolor[rgb]{0.73,0.40,0.53}{#1}}
\newcommand{\ImportTok}[1]{#1}
\newcommand{\CommentTok}[1]{\textcolor[rgb]{0.38,0.63,0.69}{\textit{#1}}}
\newcommand{\DocumentationTok}[1]{\textcolor[rgb]{0.73,0.13,0.13}{\textit{#1}}}
\newcommand{\AnnotationTok}[1]{\textcolor[rgb]{0.38,0.63,0.69}{\textbf{\textit{#1}}}}
\newcommand{\CommentVarTok}[1]{\textcolor[rgb]{0.38,0.63,0.69}{\textbf{\textit{#1}}}}
\newcommand{\OtherTok}[1]{\textcolor[rgb]{0.00,0.44,0.13}{#1}}
\newcommand{\FunctionTok}[1]{\textcolor[rgb]{0.02,0.16,0.49}{#1}}
\newcommand{\VariableTok}[1]{\textcolor[rgb]{0.10,0.09,0.49}{#1}}
\newcommand{\ControlFlowTok}[1]{\textcolor[rgb]{0.00,0.44,0.13}{\textbf{#1}}}
\newcommand{\OperatorTok}[1]{\textcolor[rgb]{0.40,0.40,0.40}{#1}}
\newcommand{\BuiltInTok}[1]{#1}
\newcommand{\ExtensionTok}[1]{#1}
\newcommand{\PreprocessorTok}[1]{\textcolor[rgb]{0.74,0.48,0.00}{#1}}
\newcommand{\AttributeTok}[1]{\textcolor[rgb]{0.49,0.56,0.16}{#1}}
\newcommand{\RegionMarkerTok}[1]{#1}
\newcommand{\InformationTok}[1]{\textcolor[rgb]{0.38,0.63,0.69}{\textbf{\textit{#1}}}}
\newcommand{\WarningTok}[1]{\textcolor[rgb]{0.38,0.63,0.69}{\textbf{\textit{#1}}}}
\newcommand{\AlertTok}[1]{\textcolor[rgb]{1.00,0.00,0.00}{\textbf{#1}}}
\newcommand{\ErrorTok}[1]{\textcolor[rgb]{1.00,0.00,0.00}{\textbf{#1}}}
\newcommand{\NormalTok}[1]{#1}


\usepackage{graphicx,grffile}
\makeatletter
\def\maxwidth{\ifdim\Gin@nat@width>\linewidth\linewidth\else\Gin@nat@width\fi}
\def\maxheight{\ifdim\Gin@nat@height>\textheight\textheight\else\Gin@nat@height\fi}
\makeatother
% Scale images if necessary, so that they will not overflow the page
% margins by default, and it is still possible to overwrite the defaults
% using explicit options in \includegraphics[width, height, ...]{}
%\setkeys{Gin}{width=\maxwidth,height=\maxheight,keepaspectratio}



\IfFileExists{parskip.sty}{%
\usepackage{parskip}
}{% else
\setlength{\parindent}{0pt}%
\setlength{\parskip}{6pt plus 2pt minus 1pt}
}

\setlength{\emergencystretch}{3em}  % prevent overfull lines
\providecommand{\tightlist}{%
  \setlength{\itemsep}{0pt}\setlength{\parskip}{0pt}}

\setcounter{secnumdepth}{0}

% Redefines (sub)paragraphs to behave more like sections
\ifx\paragraph\undefined\else
\let\oldparagraph\paragraph
\renewcommand{\paragraph}[1]{\oldparagraph{#1}\mbox{}}
\fi
\ifx\subparagraph\undefined\else
\let\oldsubparagraph\subparagraph
\renewcommand{\subparagraph}[1]{\oldsubparagraph{#1}\mbox{}}
\fi


% set default figure placement to htbp
\makeatletter
\def\fps@figure{htbp}
\makeatother

\usepackage{longtable}
\usepackage{caption}
\captionsetup[table]{position=bottom}

\usepackage[acronym, toc]{glossaries}
\usepackage{cleveref}

% Since we are including our own headers, we end up overwriting pandoc-crossref, so
% we include the commands here manually - see https://github.com/lierdakil/pandoc-crossref/issues/50
\AtBeginDocument{%
\renewcommand*\figurename{Figure}
\renewcommand*\tablename{Table}
}
\AtBeginDocument{%
\renewcommand*\listfigurename{List of Figures}
\renewcommand*\listtablename{List of Tables}
}
\usepackage{float}
\floatstyle{ruled}
\makeatletter
\@ifundefined{c@chapter}{\newfloat{codelisting}{h}{lop}}{\newfloat{codelisting}{h}{lop}[chapter]}
\makeatother
\floatname{codelisting}{Listing}
\newcommand*\listoflistings{\listof{codelisting}{List of Listings}}
